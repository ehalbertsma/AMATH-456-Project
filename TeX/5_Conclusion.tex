\newenvironment{acknowledgement}%       New acknowledgement environment
    {\large\bfseries Acknowledgements%
    \par\medskip\normalfont\normalsize}%
    {}%
\chapter{Conclusion}
\begin{multicols}{2}

Shape Memory Alloys are a very fascinating subclass of smart materials. The hallmark of these materials is the shape memory effect, which enables them to make transitions from one solid phase of matter to another while ``remembering" the original state they were in. Transitions of SMAs between the \textit{martensite} and \textit{austensite} phases can be induced through temperature changes, external applied stresses and even being subjected to the presence of a magnetic field, in the case of magnetic SMAs. 

Techniques from the calculus of variations are useful when modeling SMAs from a mathematical standpoint. Whether in a Homogenized Energy, Entropy-Gradient, or magnetic SMA model, the free energy can be formulated as a functional, the variations of which yield the constitutive equations that dictate the material behaviour. Using calculus of variations, it is possible to use these materials as primitive computers, giving them roles in impressive moments of science, such as their application in the Mars Rover.  Due to their shape memory quality, special consideration is given to the thermomechanical coupling of these materials to accurately simulate their dynamics.

We have also breifly discussed a few of the many applications for SMAs used in the aerospace industry. These included \textit{Wing Morphing} where a shape memory alloy was used to craft airplane wings in such a way as to allow them adapt to various flight conditions on their own, making for an increase in the quality of air travel, \textit{Mars Rover Tires} where SMA's were used to craft prototypes of high quality tires that can endure rough terrain on the surface of Mars by making use of the pseudoelastic effect, and \textit{Solar Sails} where SMAs were used as a deployment mechanism for the sails on this spacecraft, so that it can be propelled entirely by solar photons.
 












\begin{acknowledgement}
We are grateful to Ralph Smith for his comprehensive book on modeling smart materials. We also thank Dr. Francis Poulin for providing us with some sample codes which helped us get started. Finally, we want to thank Dr. Morris for her guidance, support and encouragement in conducting this project. 
\end{acknowledgement}
\end{multicols}








